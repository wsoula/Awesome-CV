%-------------------------------------------------------------------------------
%	SECTION TITLE
%-------------------------------------------------------------------------------
\cvsection{Work Experience}


%-------------------------------------------------------------------------------
%	CONTENT
%-------------------------------------------------------------------------------
\begin{cventries}

  \cventry
    {Sr. Specialist, Site Reliability Engineer} % Job title
    {Thread Research} % Organization
    {Remote} % Location
    {Nov. 2022 - Present} % Date(s)
    {
      \begin{cvitems} % Description(s) of tasks/responsibilities
        \item {Build dashboard for services and framework to standardize dashboards
        \item {Troubleshoot customer issues using DataDog, reducing time spent finding the issue by 1 day}
        \item {Popularize DataDog knowledge to my team of 6 people as well as customer support}
        \item {Suggested solution to reduce lambda invocations from 3000 a minute to 3}
        \item {Read through code and learned system to discover 12000 SQS messages were coming from a QA test rather than an actual client and could be removed}
        \item {Created CI/CD for dashboards and other tooling using CDKTF and BitBucket pipelines}
        \item {Web/Mobile Platform: http://threadresearch.com}
      \end{cvitems}
    }
    {
      \#Agile ~
      \#Transformation ~
      \#AWS ~
      \#Microservices ~
      \#python ~
      \#nodejs ~
    }

%---------------------------------------------------------
  \cventry
    {Sr. Site Reliability Engineer} % Job title
    {Starz} % Organization
    {Denver - Colorado} % Location
    {Sept. 2019 - Nov. 2022} % Date(s)
    {
      \begin{cvitems} % Description(s) of tasks/responsibilities
        \item {Created over 300 runbooks and alarms to standup a tier1 support team in India}
        \item {Implemented chat bot to bring data from different systems and accounts into one location and automate manual tasks}
        \item {Implemented datastore to save CloudWatch data to use for capacity planning}
        \item {Saved 10 hours weekly by automating the capacity planning process leading to at least a 10\% decrease in errors during planning}
        \item {Implemented DataDog, NewRelic, and SignalFX in search of an APM}
        \item {Created CloudWatch dashboards and alarms for all main services using the CDK}
        \item {Created pipeline in AWS for CDK pushes to GitHub to automatically deploy}
        \item {Created dashboards and alarms for services as they get implemented in SignalFX using CDKTF}
        \item {Implemented Atlantis so we can do git ops for our SignalFX IaC}
        \item {Troubleshoot issues as on call engineer}
        \item {Trying to automate myself out of a job, continuously looking to solve problems}
        \item {---}
        \item {Web/Mobile Platform: http://starz.com}
      \end{cvitems}
    }
    {
      \#Agile ~
      \#Microservices ~
      \#python ~
    }

%---------------------------------------------------------
  \cventry
    {CM / Build Engineer / Sr. Site Reliablity Engineer} % Job title
    {DrillingInfo} % Organization
    {Austin - Texas / Denver - Colorado} % Location
    {June 2012 - Sept. 2019} % Date(s)
    {
      \begin{cvitems} % Description(s) of tasks/responsibilities
        \item {Convert manual build process to Jenkins}
        \item {Implement Jenkins job builder so every job in Jenkins is in source control}
        \item {Created deploy process for chef and Cloudformation}
        \item {Implemented Cloudformation so every AWS resource was in source control}
        \item {Implemented tracing (jaeger) starting at nginx}
        \item {Implemented two monitoring platforms with the second one automatically monitoring our endpoints from across the world}
        \item {Created chatbot that quickly became used throughout the engineering department}
        \item {Developed dashboards to socialize information from disparate systems}
        \item {Contribute to opensource projects to solve our needs including: jjb, cloud- custodian, chef cookbooks, Jenkins plugins, atlantis, jaeger and more}
        \item {Operate consul and nomad at scale to deploy our containers (~200 unique svcs)}
        \item {Creating Jenkins pipeline for terraform code so changes flow to prod and switched to atlantis to make the process even easier}
        \item {Investigated rundeck and implemented container for use daily as well as deploys}
        \item {Troubleshoot issues as oncall engineer}
        \item {---}
        \item {Web Platform: http://api.krungthai-axa.co.th}
      \end{cvitems}
    }
    {
      \#Agile ~
      \#Golang ~
      \#Redis ~
      \#Rabbimq ~
      \#MongoDb ~
      \#ElasticSearch ~
      \#Microservices ~
      \#Kibana ~
      \#Nomad ~
    }

%---------------------------------------------------------
  \cventry
    {QA Analyst / CM} % Job title
    {mobicorp} % Organization
    {Austin - Texas} % Location
    {Sept. 2005 - June 2012} % Date(s)
    {
      \begin{cvitems} % Description(s) of tasks/responsibilities
        \item {Reduced regressions by designing and implementing automated regression test and integrating it into the nightlybuild process}
        \item {Read about WebTest on the internet, was impressed with it and ported entire web GUI automated regression suite to WebTest}
        \item {Managed intern for automated regression test writing}
        \item {Sole tester of the optimizing engine with XML, batch files, diff}
        \item {Implemented automated integration tests of the optimization engine in Jenkins}
        \item {On my own initiative learned Hudson and installed plugin to email test results and SVN commits from nightlybuilds.}
        \item {Decreased time from SVN checkin to WebTest completion by several hours through increased use of slave machines}
        \item {Managed/created the continuous integration build process}
        \item {Implemented Cobertura code coverage}
        \item {Created three installers and maintained three additional ones}
        \item {Created promotion process where QA deployed the latest code from Jenkins and that became the production build}
      \end{cvitems}
    }
    {
      \#Agile ~
      \#Jenkins ~
    }

\end{cventries}
